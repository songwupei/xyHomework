\documentclass[a4paper, 14pt]{article}
\usepackage{xydailystudy}

\begin{document}

% 标题部分
\dailytitle{2025年10月14日}

% 识字学习反馈部分
\begin{hanzibox}
\par
\textbf{一、今日课程内容与目标}

\begin{enumerate}
\item \textbf{听写复习:}听写上节课汉字"四时,天空,地下,早上,中午,夜晚",并改错。

\item \textbf{新字学习:}学习书写汉字:"金,木,水,火,土",掌握正确笔顺。

\item \textbf{词语诵读:}鼓励孩子诵读"泥土,土豆,树木,木材,开水,水果,点火,金鱼"等词语。

\item \textbf{笔画练习:}基本笔画练习:"横折"。

\item \textbf{韵文学习:}熟读"对韵歌"。
\end{enumerate}

\textbf{二、温馨提示}

\begin{enumerate}
\item \textbf{注意事项:}汉字书写笔顺及占格。

\item \textbf{课后作业:}完成练习四、五(如下图),请家长协助孩子合理安排时间。背诵"对韵歌"。

\item \textbf{作业要求:}作业尽力即可,每个孩子学习进度不同,请根据孩子能力完成,下节课将听写上节课所学汉字。

\item \textbf{学习记录:}欢迎自愿打卡记录学习过程,共同鼓励孩子坚持练习。
\end{enumerate}

感谢各位家长的配合与支持,我们一起助力小朋友快乐学汉字!\texttwemoji{1f33c}

\end{hanzibox}

% 英语学习反馈部分
\begin{englishbox}
\par
\textbf{Englisi Reading Class Report: \textit{One Teddy Bear All Alone}}

\textbf{一、热身与复习}

我们以活跃的-all词族复习开始课程。我们练习阅读单词如ball, call, fall, hall, mall, tall, wall和small,以加强发音和拼写模式。

\textbf{二、新视觉词与词汇}

我们介绍了本故事的新视觉词:one, two。

我们还学习了新的图片词汇来帮助理解故事:teddy bear, phone, kite, tree, sea。

主要练习的结构是:"all alone"和"all for me"。

\textbf{三、主要故事}

我们一起阅读故事,跟随泰迪熊的冒险一起数数。孩子们喜欢迷人的图片,并享受预测泰迪熊接下来会做什么。简单重复的模式有助于建立阅读信心。

关键结构:重复的计数模式,例如"One teddy bear all alone","Two teddy bears on the phone"。

\textbf{四、今日小任务}

\begin{enumerate}
\item 请练习阅读和书写视觉词:one, two。
\item 鼓励您的孩子大声朗读故事给您听,指着每个单词并使用图片帮助理解。
\end{enumerate}

祝您度过一个愉快的夜晚!

\end{englishbox}

% 作业记录部分 - 只显示识字和英语作业
\homeworkrecord{}{请练习阅读和书写视觉词:one, two;\\鼓励孩子大声朗读故事《One Teddy Bear All Alone》}{完成练习四、五;\\背诵"对韵歌"}{}

\end{document}