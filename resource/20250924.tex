\documentclass[a4paper, 14pt]{article}
\usepackage{xydailystudy}

\begin{document}

% 标题部分
\dailytitle{2025年9月24日}

% 拼音学习反馈部分
\begin{pinyinbox}
\par
\textbf{一、学习内容}

\begin{enumerate}
\item \textbf{声母f的认读:}上齿接触下唇,形成窄缝,气流从中发出,气音短促!助记忆儿歌:\\
扎起头发fff/一根拐杖fff

\item \textbf{书写:}f占中上格,两笔写成。横比较短,写在第二条线略低一点的位置。

\item \textbf{拼读练习(课堂书写):}
\begin{itemize}
\item f-a → fa fā fá fǎ fà
\item f-o → fo fó
\item f-u → fu fū fú fǔ fù
\end{itemize}

\item \textbf{听写:}bō、bà、pí 、mò、mí
\end{enumerate}

\textbf{二、今日小任务}

\begin{enumerate}
\item 熟读声母f+韵母aou的拼读及组词,能够发音准确
\item 如图,幼小衔接测试卷单韵母i、u、ü的练习
\end{enumerate}

\textbf{三、温馨提示:}后续课堂将逐步增加听写频次,建议在家练习时可模拟课堂听写模式:家长读拼音(如"bí"),孩子先口头重复确认,再规范书写,写完后对照正确答案自查。通过反复熟悉"听—认—写—查"的流程,帮助孩子减少听写时的紧张感,提升书写准确性和反应速度。

\end{pinyinbox}

% 作业记录部分 - 只显示拼音作业
\homeworkrecord{熟读声母f+韵母aou的拼读及组词,能够发音准确;\\幼小衔接测试卷单韵母i、u、ü的练习}{}{}{}

\end{document}
